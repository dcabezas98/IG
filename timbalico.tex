\documentclass[12pt]{article}
\usepackage[left=3cm,right=3cm,top=2cm,bottom=2cm]{geometry} % page
                                                             % settings
\usepackage{amsmath} % provides many mathematical environments & tools
\usepackage[spanish]{babel}
\usepackage[doument]{ragged2e}

% Images
\usepackage{graphicx}
\usepackage{float}

% Code
\usepackage{listings}
\usepackage{xcolor}
\definecolor{gray}{rgb}{0.5,0.5,0.5}
\newcommand{\n}[1]{{\color{gray}#1}}
\lstset{numbers=left,numberstyle=\small\color{gray}}

\selectlanguage{spanish}
\usepackage[utf8]{inputenc}
\setlength{\parindent}{0mm}

\usepackage{enumerate}

\begin{document}

\title{Timbálico: Grafo PHIGS}
\author{David Cabezas}
\date{\today}
\maketitle

He creado un modelo jerárquico con múltiples componentes que se mueven
de diversas formas, lo he bautizado como Timbálico. Consta de un
soporte con cuatro brazos, cada uno con una hélice que se desplaza por él
a la vez que gira y una pelota en el extremo que se infla y desinfla.

\subsection*{Grafo PHIGS}
\vspace{-5mm}

\begin{figure}[H]
 \hspace{-30mm} \includegraphics[width=200mm]{PHIGS-timbalico}
\end{figure}

\subsection*{Parámetros / Grados de libertad}

El Timbálico tiene 14 grados de libertad, que detallo a continuación:
\begin{enumerate}

\item \textbf{Traslación del Timbálico:}
  El Timbálico completo sube y baja (oscila en el eje $Y$) entre las alturas 1 y -1.
  \\ $e=\sin(t_{sec}\pi/10)$
\item \textbf{Rotación del Timbálico:}
  Los brazos del Timbálico rotan sobre el eje $Y$, alrededor del soporte. \\ $d=90t_{sec}$
\item \textbf{Rotación y traslación de la PelotaInflable 1:}
  La PelotaInflable 1 se agranda y empequeñece, su radio oscila entre 1 y 2 y su centro se desplaza para que su superficie siempre quede en el extremo del Eje del Brazo 1. \\ $a=7.5+0.5\sin(t_{sec}\pi/2)$
\item \textbf{Rotación y traslación de la PelotaInflable 2:}
  Análogo a la rotación y traslación de la PelotaInflable 1, pero en el Brazo 2.
\item \textbf{Rotación y traslación de la PelotaInflable 3:}
  Análogo a la rotación y traslación de la PelotaInflable 1, pero en el Brazo 3.
\item \textbf{Rotación y traslación de la PelotaInflable 4:}
  Análogo a la rotación y traslación de la PelotaInflable 1, pero en el Brazo 4.
\item \textbf{Rotación de la Helice 1:}
  La Helice 1 rota sobre el Eje del Brazo 1 (inicialmente el eje $X$).
  \\ $b=360t_{sec}$
\item \textbf{Rotación de la Helice 2:}
  Análogo a la rotación de la Helice 1, pero sobre el Brazo 2.
\item \textbf{Rotación de la Helice 3:}
  Análogo a la rotación de la Helice 1, pero sobre el Brazo 3.
\item \textbf{Rotación de la Helice 4:}
  Análogo a la rotación de la Helice 1, pero sobre el Brazo 4.
\item \textbf{Traslación de la Helice 1:} La Helice 1 recorre el Eje
  del Brazo 1 (inicialmente el eje $X$), oscilando entre el Soporte y
  la PelotaInflable1. \\ $c=3+2.9\sin(t_{sec}\pi/4)$
\item \textbf{Traslación de la Helice 2:}
  Análogo a la traslación de la Helice 1, pero sobre el Brazo 2.
\item \textbf{Traslación de la Helice 3:}
  Análogo a la traslación de la Helice 1, pero sobre el Brazo 3.
\item \textbf{Traslación de la Helice 4:}
  Análogo a la traslación de la Helice 1, pero sobre el Brazo 4.
\end{enumerate}

\subsection*{Nodos terminales}

Sólo he necesitado dos tipos de mallas indexadas para los nodos terminales. \\

Una es la Esfera de revolución que hice en la práctica 2, la he pintado de rojo. \\

El otro es CilindroCerrado, una modificación del cilindro de
revolución, le he puesto tapas y lo he centrado en el origen (las
bases están a las alturas $y=-0.5$ e $y=0.5$). Lo he transformado para
el soporte (pintado de marrón) y para los ejes (pintado de
azul). También he creado CilindroCerrados con 5 perfiles (sale
un prisma cuadrangular) para las hélices, que he pintado de amarillo.

\end{document}
